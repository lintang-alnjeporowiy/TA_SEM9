\chapter{LAMPIRAN}
\label{chap:lampiran}

\section*{TES 123}
\label{sec:tes-lampiran}

% \begin{enumerate}[label=Lampiran \arabic*]
%     \item
% \end{enumerate}

\section*{Hasil Kuisioner}
Berikut adalah hasil kuisioner yang digunakan dalam penelitian:

\begin{table}[htbp]
    \centering
    \begin{tabular}{|c|l|c|}
        \hline
        No & Pertanyaan & Jumlah Responden \\
        \hline
        1  & Apakah Anda menggunakan layanan kami? & 120 \\
        2  & Seberapa puas Anda dengan layanan kami? & 100 \\
        3  & Apakah Anda merekomendasikan layanan kami? & 110 \\
        \hline
    \end{tabular}
    \caption{Hasil Kuisioner Responden}
    \label{tab:kuisioner}
\end{table}

\section*{Gambar atau Diagram Pendukung}
Gambar berikut menunjukkan diagram proses yang digunakan dalam penelitian:

% \begin{figure}[htbp]
%     \centering
%     \includegraphics[width=0.8\textwidth]{gambar/diagram.png} % Ganti dengan path file gambar
%     \caption{Diagram Proses Penelitian}
%     \label{fig:diagram}
% \end{figure}

\section*{Kode Program}
Kode program berikut digunakan untuk menganalisis data:

\begin{verbatim}
# Python code for data analysis
import pandas as pd

data = pd.read_csv('data.csv')
print(data.describe())
\end{verbatim}

\section*{Dokumen Lainnya}
Berikut adalah dokumen lain yang relevan untuk penelitian:
\begin{itemize}
    \item Surat izin penelitian
    \item Contoh kuisioner
    \item Bukti pengumpulan data
\end{itemize}



