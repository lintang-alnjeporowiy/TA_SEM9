\begin{center}
  \Large
  \textbf{KATA PENGANTAR}
\end{center}

\vspace{2ex}

Puji dan syukur kehadirat Allah SWT atas segala limpahan berkah, rahmat, serta hidayah-Nya, penulis  dapat menyelesaikan penelitian ini dengan judul
``\textbf{Desain Konseptual Kapal Pengangkut BBM untuk Wilayah Kepulauan 3T: Studi Kasus Kabupaten Maluku Barat Daya}''.
Penelitian ini disusun dalam rangka pemenuhan bidang riset di Departemen Teknik Transportasi Laut,
  serta digunakan sebagai persyaratan menyelesaikan pendidikan S1.

Dalam penyusunan buku ini,
  penulis mengucapkan terima kasih kepada Keluarga yang telah memberikan dorongan spiritual dan material dalam penyelesaian penelitian ini.
Terutama kepada Ibu atas dukungannya kepada penulis, Terimakasih telah sabar dan tak pernah lelah memberikan semangat.

Penulis juga mengucapkan terima kasih kepada Bapak Ir. Tri Achmadi, Ph.D 
  dan Muhammad Riduwan, S.Kom., M.Kom atas arahan dan bimbingan selama pengerjaan penelitian tugas akhir ini.
Serta kepada Bapak-ibu dosen pengajar Departemen Teknik Transportasi Laut atas pengajaran dan perhatian yang diberikan kepada penulis selama ini.

Dan terakhir,
  terima kasih kepada rekan-rekan Omah Prestasi dan Kos Qur'an 9 atas pengalamannya kepada penulis.
Serta kepada rekan-rekan seperjuangan Teknik Transportasi Laut 2020, AKSAVANA T-18.

Kesempurnaan hanya milik Allah SWT, untuk itu penulis memohon segenap kritik dan saran yang  membangun.
Semoga penelitian ini dapat memberikan manfaat bagi kita semua, aamiin.

\vspace{4ex}

\begin{flushright}
  \begin{tabular}[b]{c}
    Surabaya, Juli 2024\\
    \\
    \\
    \\
    \\
    Lintang Al Hilal Fitri
  \end{tabular}
\end{flushright}
