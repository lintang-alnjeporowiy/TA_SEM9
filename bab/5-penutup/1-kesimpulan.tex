\section{Kesimpulan}
\label{sec:kesimpulan}

\begin{enumerate}
    \item Hasil simulasi persediaan BBM menunjukkan moda transportasi laut saat ini belum mampu memenuhi permintaan BBM Kabupaten Maluku Barat Daya dengan ekspektasi rata-rata permintaaan yang tidak terpenuhi selama satu tahun untuk bensin sebesar 24,79 kL, minyak tanah sebesar 112 kL dan solar sebesar 1,11 kL.
    \item Moda transportasi laut yang dirancang  memiliki panjang sebesar 53,42 meter, lebar 8,52 meter, tinggi 4,11 meter, dan sarat air sebesar 3,17 meter. Kecepatan kapal adalah 7,75 knot. Kapal ini dilengkapi dengan tangki bensin berkapasitas 252,72 kL, tangki solar dengan kapasitas 202,7 kL, dan tangki minyak tanah yang berkapasitas 396,74 kL
    \item Potensi penghematan jika menggunakan sistem baru yaitu dengan sistem pengiriman \emph{multiport} dan penambahan kapasitas penyimpanan sebesar 280 juta rupiah pertahun
\end{enumerate}