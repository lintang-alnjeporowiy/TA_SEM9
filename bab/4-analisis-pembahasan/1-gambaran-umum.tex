\section{Gambaran Umum}
\label{sec:gambaran-umum}

Kabupaten Maluku Barat Daya adalah salah satu kabupaten dari Provinsi Maluku. Lokasinya yang jauh dari pusat provinsi (Kota Ambon) dan lebih dekat ke Timor Leste membuatnya tercakup dalam daerah 3T (Tertinggal, Terdepan dan Terluar). Jumlah penduduk Kabupaten MBD sekitar 91.387 jiwa dengan PDRB per kapita sebesar 23,96 Juta per tahun.

\begin{figure}[ht!]
    \centering
    \includegraphics[width=0.8\textwidth]{gambar/administrasi-maluku-barat-daya-a1.jpg}
    \caption{Peta Administrasi Kabupaten Maluku Barat Daya \citep{Peta_Tematik_Indonesia_2014}}
    \label{fig:peta-mbd-basic}
\end{figure}

\subsection{Model Permintaan BBM Kabupaten Maluku Barat Daya}
\label{subsec:variasi-konsumsi-bbm-day}

Kondisi angka konsumsi harian BBM yang tidak tetap akan dipotret dengan cara memodelkannya menjadi sebuah distribusi kumulatif. Data yang digunakan adalah data realisasi pemasokan BBM pada tahun 2021 sebagaimana yang ditampilkan pada gambar \ref{fig:realisasi-bbm-mbd}. Kemudian dikalikan dengan persentase konsumsi tiap titiknya yang diasumsikan \emph{flat} sesuai dengan volume BBM yang dipasok disetiap titiknya sebagaimana gambar \ref{fig:grafik-rasio-pskan-antar-daerah}.

\begin{figure}[ht!]
    \centering
    \includegraphics[width=0.8\textwidth]{gambar/rasio-psokan-bbm-mbd.png}
    \caption{Grafik Pemasokan BBM Kabupaten Maluku Barat Daya berdasarkan Titik Pemasokan}
    \label{fig:grafik-rasio-pskan-antar-daerah}
\end{figure}

Hasil pengolahan dan parameter variasi input dapat dilihat pada tabel \ref{tabel-data-variasi-input}. Fungsi distribusi yang digunakan ada dua jenis. Fungsi \emph{Pert} memberikan luaran berupa kurva normal yang biasa dengan parameter \emph{minimum, most likely} dan \emph{maximum}. Fungsi \emph{PertAlt} memberikan luaran yang sama dengan \emph{Pert} namun diberikan pilihan untuk mengatur letak tiap nilai melaui persentil yang diinginkan. Fungsi\emph{PertAlt} digunakan untuk memodelkan konsumsi solar dan bensin karena bentuk distribusi yang condong ke nilai yang lebih kecil dengan dasaran nilai realiasi konsumsi yang kecil. Konsumsi minyak tanah dimodelkan dengan fungsi \emph{Pert} karena keyakinan bahwa angka konsumsi harian tidak berbeda jauh dengan hasil estimasi sesuai data pasokan dan realisasi BBM Kabupaten Maluku Barat Daya.

\begin{table}[!ht]
    \centering
    \caption{Tabel Variasi Input untuk Konsumsi Harian BBM}
    \begin{tabular}{|c|c|c|c|c|c|}
    \hline
        \textbf{Daerah} & \textbf{Komoditas} & \textbf{Fungsi Distribusi} & \textbf{Minimal [kL]} & \textbf{Most Likely [kL]} & \textbf{Max [kL]} \\ \hline
        \textbf{} & Bensin & PertAlt & 0.020 & 0.023 & 0.256 \\ \hline
        \textbf{Lakor} & Solar & PertAlt & 0.009 & 0.012 & 0.116 \\ \hline
        \textbf{} & Minyak Tanah & Pert & 0.349 & 0.410 & 0.451 \\ \hline
        \textbf{} & Bensin & PertAlt & 0.021 & 0.025 & 0.273 \\ \hline
        \textbf{Letti} & Solar & PertAlt & 0.010 & 0.012 & 0.124 \\ \hline
        \textbf{} & Minyak Tanah & Pert & 0.372 & 0.437 & 0.481 \\ \hline
        \textbf{} & Bensin & PertAlt & 0.020 & 0.023 & 0.256 \\ \hline
        \textbf{Romang} & Solar & PertAlt & 0.009 & 0.012 & 0.116 \\ \hline
        \textbf{} & Minyak Tanah & Pert & 0.349 & 0.410 & 0.451 \\ \hline
        \textbf{} & Bensin & PertAlt & 0.237 & 0.279 & 3.067 \\ \hline
        \textbf{Tiakur} & Solar & PertAlt & 0.074 & 0.093 & 0.930 \\ \hline
        \textbf{} & Minyak Tanah & Pert & 2.789 & 3.281 & 3.609 \\ \hline
        \textbf{} & Bensin & PertAlt & 0.137 & 0.162 & 1.779 \\ \hline
        \textbf{Tepa} & Solar & PertAlt & 0.043 & 0.054 & 0.539 \\ \hline
        \textbf{} & Minyak Tanah & Pert & 1.618 & 1.903 & 2.093 \\ \hline
    \end{tabular}
    \label{tabel-data-variasi-input}
\end{table}

\begin{figure}[htbp]
    \centering
    \begin{subfigure}[b]{0.48\textwidth}
        \centering
        \includegraphics[width=\textwidth]{gambar/cons-bbm-tepa.png}
        \caption{Tepa}
        \label{fig:cons-bbm-tepa}
    \end{subfigure}
    \hfill
    \begin{subfigure}[b]{0.48\textwidth}
        \centering
        \includegraphics[width=\textwidth]{gambar/cons-bbm-lakor.png}
        \caption{Lakor}
        \label{fig:cons-bbm-lakor}
    \end{subfigure}
    
    \vspace{1em}
    
    \begin{subfigure}[b]{0.48\textwidth}
        \centering
        \includegraphics[width=\textwidth]{gambar/cons-bbm-letti.png}
        \caption{Letti}
        \label{fig:cons-bbm-letti}
    \end{subfigure}
    \hfill
    \begin{subfigure}[b]{0.48\textwidth}
        \centering
        \includegraphics[width=\textwidth]{gambar/cons-bbm-romang.png}
        \caption{Romang}
        \label{fig:cons-bbm-romang}
    \end{subfigure}
    
    \vspace{1em}
    
    \begin{subfigure}[b]{0.48\textwidth}
        \centering
        \includegraphics[width=\textwidth]{gambar/cons-bbm-romang.png}
        \caption{Tiakur}
        \label{fig:cons-bbm-tiakur}
    \end{subfigure}
    
    \caption{Grafik Distribusi Kumulatif Konsumsi Harian BBM}
    \label{fig:all-bbm-graphs}
\end{figure}

\subsection{Sistem Penyaluran BBM Kabupaten Maluku Barat Daya}
\label{subsec:sistem-eksisting}

    Kebutuhan BBM Kabupaten Maluku Barat Daya dilayani oleh Terminal Bahan Bakar Minyak Saumlaki didalam cakupan PT. Pertamina Patra Niaga Regional Maluku Papua. Berdasarkan SK Kepala BPH Migas No 55 Tahun 2019 dan realisasi pemasokan yang dilakukan oleh PT. Pertamina (Persero) didapatkan data pasokan BBM seperti gambar \ref{fig:realisasi-bbm-mbd}.

\begin{figure}[htbp!]
    \centering
    \includegraphics[width=0.8\textwidth]{gambar/pasokan-konsumsi-BBM-MBD.png}
    \caption{Grafik Kuota dan Realisasi BBM Kabupaten Maluku Barat Daya Tahun 2021 \citep{Pertamina_2021}}
    \label{fig:realisasi-bbm-mbd}
\end{figure}

    Pemasokan BBM Kabupaten MBD saat ini dilayani oleh tiga kapal yaitu SPOB Handil Tirusan, SPOB Cinta Damai dan LCT Buma III.
    Sistem penyaluran bersifat \emph{direct} dari TBBM Saumlaki menuju 5 titik pemasokan yaitu Tiakur, Tepa, Lakor, Letti dan Romang. Ilustrasi rute dapat dilihat pada gambar \ref{fig:rute-old-bbm-mbd}. Warna merah muda melambangkan kapal SPOB Handil Tirusan, Hijau adalah SPOB Cinta Damai dan Biru adalah LCT BUMA III.

\begin{figure}[htbp!]
    \centering
    \includegraphics[width=0.8\textwidth]{gambar/rute-eksisting.png}
    \caption{Pola Operasi Pemasokan BBM Kabupaten MBD Saat Ini}
    \label{fig:rute-old-bbm-mbd}
\end{figure}

    SPOB Cinta Damai melayani tiga titik pasokan yaitu Letti, Romang dan Lakor dengan rata-rata muatan 75 kiloliter. SPOB Handil Tirusan melayani titik Tiakur dengan rata-rata muatan 50 kiloliter. LCT BUMA III melayani titik Tepa dengan rata-rata muatan 50 kiloliter. Ringkasan penyaluran BBM berdasarkan moda kapal dapat dilihat pada gambar 

\begin{figure}[htbp!]
    \centering
    \includegraphics[width=0.8\textwidth]{gambar/pemasokan-BBM-MBD.png}
    \caption{Grafik Pemasokan BBM Kabupaten Maluku Barat Daya Tahun 2023 berdasarkan kapal}
    \label{fig:kapal-mbd-old}
\end{figure}

Sistem pemasokan ini masih menggunakan kapal jenis SPOB \emph{(Self Propelled Oil Barge)} dan LCT \emph{(Landing Craft Tank)} dimana kedua karakteristik kedua jenis kapal tersebut mempunyai ketinggian \emph{freeboard} yang rendah. Konsekuensi \emph{freeboard} yang rendah ini akan membatasi cuaca dimana kapal ini dapat beroperasi. Batasan operasional dan dampaknya terhadap pemasokan BBM akan dibahas pada bab selanjutnya.
