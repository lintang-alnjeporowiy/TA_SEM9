\section{Analisis Kesenjangan}
\label{sec:analisis-kesenjangan}

Perbandingan biaya yang dikeluarkan antara kondisi yang ada dengan sistem penyaluran baru yang diusulkan dapat dilihat pada tabel \ref{tab:perbandingan-biaya}. Perbedaan struktur biaya terlihat jelas dimana biaya tetap yang harus dikeluarkan untuk sistem baru lebih besar, Namun jika dilihat biaya BBM yang dibutuhkan jauh lebih sedikit. Hal ini merupakan efek langsung dari pengurangan frekuensi pemasokan. Potensi penghematan yang bisa didapatkan sebesar 5,3 miliar rupiah, jika berganti ke sistem pemasokan baru. 

\begin{table}[htbp]
\centering
\begin{tabular}{|l|r|r|r|}
\hline
& \multicolumn{1}{c|}{Biaya BBM} & \multicolumn{1}{c|}{Biaya Tangki Timbun} & \multicolumn{1}{c|}{Biaya Tetap} \\
\hline
Kondisi Saat Ini & 10.552,91 & 0,00 & 1.229,81 \\
Usulan Baru & 322,29 & 75,20 & 6.074,63 \\
\hline
\end{tabular}
\caption{Perbandingan Biaya (dalam juta rupiah)}
\label{tab:perbandingan-biaya}
\end{table}

Sistem baru walaupun \emph{load factor} rata-rata masih jauh dari kata optimum karena ketimpangan konsumsi antar jenis BBM. Namun, jika dilihat masih lebih tinggi dibandingkan dengan kapal yang saat ini digunakan. Perbandingan operasional dapat dilihat pada tabel \ref{tab:operational-gap} 


\begin{table}[htbp]
\centering
\begin{tabular}{|l|c|r|}
\hline
\textbf{Nama Kapal} & \textbf{Frekuensi} & \textbf{Avg. Load Factor} \\
\hline
Handil Tirusan & 40 & 33,33\% \\
BUMA III & 24 & 16,67\% \\
Cinta Damai 1 & 12 & 22,86\% \\
Sistem Baru & 9 & 59,32\% \\
\hline
\end{tabular}
\caption{Perbandungan Operasional}
\label{tab:operational-gap}
\end{table}