\section{Pengujian Hipotesis}
\label{sec:uji-hipotesis}

Hipotesis penelitian ini adalah bahwa terjadinya kelangkaan BBM di Kabupaten Maluku Barat Daya disebabkan karena desain moda pemasokan BBM yang sudah ada kurang optimal. Hipotesis ini perlu dibuktikan agar penyelesaian yang diusulkan sesuai dengan permasalahan yang terjadi. Uji hipotesis ini akan memanfaatkan dua metode yaitu uji gerakan kapal dan metode simulasi monte-carlo.

\subsection{Analisis \emph{Seakeeping} dan Stabilitas}
\label{subsec:analisis-seakeeping}

Langkah pertama dalam menguji hipotetis ini adalah menentukan batasan operasional kapal terhadap cuaca. Parameter yang digunakan adalah kemampuan olah gerak kapal dalam kondisi ketinggian gelombang tertentu. Gerakan kapal yang dilihat adalah \textit{itch, roll} dan \emph{heave}. Salah satu kapal yang sudah ada kemudian akan disimulasikan dan diuji gerakannya apakah memenuhi standar \emph{seakeeping} yang aman.4

Pengujian akan dimulai dengan memodelkan lambung kapal yang sudah ada dengan bantuan perangkat lunak \emph{Maxsurf}. Parameter selanjutnya yang diperlukan adalah \emph{range} gelombang yang digunakan. Ketinggian gelombang tersebut kemudian dimodelkan menggunakan \emph{spectra} gelombang yang sudah disediakan oleh \emph{Maxsurf Motion}. Akhirnya dilakukan simulasi dan perhitungan untuk mengetahui batasan gerak kapal yang diuji.

\subsection{Simulasi Pemasokan BBM}
\label{subsec:metode-simul-bbm}


\subsection{Penentuan Biaya Publik}
\label{subsec:awal-biaya-publik}
