\section{Tahap Persiapan}
\label{sec:tahap-persiapan}
Seperti yang ditampilkan pada gambar \ref{fig:diagramkerja} penelitian ini dimulai dengan perumusan masalah yang menjadi fokus utama pengerjaan tugas akhir. Perumusan masalah ini dilakukan melalui diskusi intensif dengan dosen pembimbing untuk mendapatkan arahan dan masukan yang tepat. Setelah permasalahan dirumuskan secara matang, langkah selanjutnya adalah melakukan studi literatur yang mendalam. Tujuannya adalah untuk mempelajari berbagai metode dan teori yang relevan dengan permasalahan yang dihadapi. Hal ini diharapkan dapat memberikan landasan yang kuat dalam menyelesaikan penelitian dan menghasilkan karya tulis yang berkualitas.
    
Selain studi literatur, penulis juga berencana melakukan studi lapangan. Studi lapangan dilakukan untuk memperoleh pemahaman yang lebih mendalam tentang kondisi lapangan permasalahan yang dihadapi. Melalui observasi langsung dan pengumpulan data primer, diharapkan dapat diperoleh gambaran yang lebih jelas dan akurat mengenai situasi dan kondisi yang sebenarnya di lapangan. Namun setelah melakukan beberapa korespondensi dengan beberapa instansi terkait, studi lapangan ini tidak dapat dilakasanakan, sehingga penelitian ini dicukupkan dengan data sekunder yang dapat diperoleh.

Rumusan masalah penelitian dan hipotesis awal yang menjadi landasan penelitian ini dilakasanakan disusun pada tahap ini. Perumusan kedua hal tersebut dilakukan dengan diskusi bersama dosen pembimbing. Setelah muncul hipotesis dan rumusan masalah yang akan diangkat diperlukan batsaan masalah agar penelitian dapat lebih terfokus dan dapat dikerjakan secara maksimal.