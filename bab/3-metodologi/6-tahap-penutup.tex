\section{Tahap Penutup}
\label{sec:tahap-penutup}

    Tahap terakhir dari tugas akhir ini setelah merancang kapal adalah melakukan analisis kesenjangan dan sensitivitas dari model sistem penyaluran BBM baru yang dirancang. Analisis kesenjangan atau \emph{gap analysis} dilakukan dengan cara membandingkan dan mencari selisih antara biaya tahunan kondisi saat ini dengan sistem baru yang usulkan.

    Selain biaya tahunan faktor lain yang dibandingkan adalah, \emph{load factor} kapal dan frekuensi trip per tahun. Hal ini dilakukan untuk mengetahui karakteristik dari masing-masing sistem penyaluran BBM.

    Analisis kedua yang dilakukan adalah sensitivitas untuk mengetahui interaksi antar variabel. Variabel yang akan dianalisis adalah waktu maksimal durasi gangguan cuaca dan bentuk distribusi yang digunakan dan pengaruhnya terhadap biaya, dan utamanya terhadap potensi kelangkaan BBM.

    Penelitian ini akan ditutup dengan kesimpulan yang tersusun atas spesifikasi kapal yang dirancang, hasil analisis kesenjangan dan analisis sensitivitas. Kesimpulan tersebut sekaligus menjadi rekomendasi untuk peningkatan sistem penyaluran BBM.