\begin{center}
  \large\textbf{ABSTRAK}
\end{center}

\vspace{2ex}

\begingroup
  \setlength{\tabcolsep}{0pt}
  \noindent
  \begin{tabularx}{\textwidth}{l >{\centering}m{2em} X}
    Nama        &:& Lintang Al Hilal Fitri \\
    Judul       &:&	Desain Konseptual Kapal Pengangkut BBM untuk Wilayah Kepulauan 3T: Studi Kasus Kabupaten Maluku Barat Daya \\
    Pembimbing  &:& 1. Ir. Tri Achmadi, Ph.D.  \\
                & & 2. Ir. Muhammad Riduwan, S.Kom., M.Kom \\
  \end{tabularx}
\endgroup

Penelitian ini bertujuan untuk mengatasi kelangkaan bahan bakar minyak (BBM) di Kabupaten Maluku Barat Daya (MBD), sebuah daerah kepulauan terluar di Indonesia. Tantangan distribusi muncul akibat gangguan cuaca yang sering menghambat pasokan BBM. Penelitian ini mengeksplorasi solusi melalui perancangan pola operasi dan moda transportasi laut yang efisien. Pola operasi akan ditentukan berdasarkan analisis kebutuhan distribusi, sedangkan desain konseptual moda transportasi laut disesuaikan dengan kondisi lingkungan Kabupaten MBD. Kapal yang sudah ada dievaluasi dari segi stabilitas dan kemampuan \emph{seakeeping}, dengan hasil evaluasi tersebut dijadikan kriteria dasar untuk merancang kapal baru. Kapal baru dirancang menggunakan spesifikasi yang dihasilkan dari simulasi Monte Carlo, yang mempertimbangkan ketidakpastian permintaan BBM dan durasi gangguan cuaca, sehingga dapat memastikan distribusi yang andal dalam berbagai skenario. Hasil penelitian ini diharapkan menjadi dasar untuk meningkatkan ketersediaan BBM dan infrastruktur transportasi laut, serta mendukung pertumbuhan ekonomi berkelanjutan di Kabupaten Maluku Barat Daya.

Kata Kunci: Desain Kapal, Distribusi BBM, Transportasi Laut
