\begin{center}
  \large\textbf{ABSTRAK}
\end{center}

\vspace{2ex}

\begingroup
  \setlength{\tabcolsep}{0pt}
  \noindent
  \begin{tabularx}{\textwidth}{l >{\centering}m{2em} X}
    Nama        &:& Lintang Al Hilal Fitri \\
    Judul       &:&	Desain Konseptual Kapal Pengangkut BBM untuk Wilayah Kepulauan 3T: Studi Kasus Kabupaten Maluku Barat Daya \\
    Pembimbing  &:& 1. Ir. Muhammad Riduwan, S.Kom., M.Kom   \\
                & & 2. Ir. Tri Achmadi, Ph.D. \\
  \end{tabularx}
\endgroup

Penelitian ini bertujuan untuk mengatasi kelangkaan bahan bakar minyak (BBM) di Kabupaten Maluku Barat Daya (MBD), sebuah daerah kepulauan terluar di Indonesia. Menghadapi tantangan distribusi yang kompleks, penelitian ini akan mengeksplorasi solusi melalui perancangan pola operasi dan moda transportasi laut yang efisien. Penelitian ini akan mencari pola operasi untuk penyaluran BBM dan merancang desain konseptual moda transportasi laut yang tepat untuk kondisi lingkungan Kabupaten MBDD. Diharapkan hasil dari penelitian ini dapat menjadi landasan bagi peningkatan ketersediaan BBM dan infrastruktur transportasi laut, serta mendukung pertumbuhan ekonomi yang berkelanjutan di Kabupaten Maluku Barat Daya.

Kata Kunci: Infrastuktur, Distribusi BBM, Transportasi Laut
