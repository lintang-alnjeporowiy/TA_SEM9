\begin{center}
  \large\textbf{ABSTRACT}
\end{center}

\vspace{2ex}

\begingroup
  \setlength{\tabcolsep}{0pt}
  \noindent
  \begin{tabularx}{\textwidth}{l >{\centering}m{2em} X}
    \emph{Name}     &:& Lintang Al Hilal Fitri \\
    \emph{Title}    &:&	\emph{"Conceptual Design of a Fuel Transport Vessel for the 3T Archipelago Areas: Case Study of Southwest Maluku Regency"} \\
    \emph{Advisors} &:& 1. Ir. Tri Achmadi, Ph.D.  \\
                    & & 2. Ir. Muhammad Riduwan, S.Kom., M.Kom \\
  \end{tabularx}
\endgroup


This study aims to address the scarcity of fuel oil (BBM) in Maluku Barat Daya (MBD) Regency, a remote archipelagic region in Indonesia. Distribution challenges arise due to frequent supply disruptions caused by adverse weather conditions. This research explores solutions through the design of efficient operational patterns and maritime transportation modes. The operational patterns will be determined based on distribution needs analysis, while the conceptual design of maritime transportation modes will be tailored to the environmental conditions of MBD Regency. Existing vessels will be evaluated in terms of stability and seakeeping performance, with the evaluation results serving as the baseline criteria for designing new vessels. The new vessels will be designed based on specifications derived from Monte Carlo simulations, which consider the uncertainties in BBM demand and weather disruption duration, ensuring reliable distribution across various scenarios. The results of this study are expected to provide a foundation for improving BBM availability and maritime transportation infrastructure, thereby supporting sustainable economic growth in Maluku Barat Daya Regency.


\emph{Keywords}: \emph{Ship Design}, \emph{Fuel Distribution}, \emph{Marine Transportation}.
