\begin{center}
  \large\textbf{ABSTRACT}
\end{center}

\vspace{2ex}

\begingroup
  \setlength{\tabcolsep}{0pt}
  \noindent
  \begin{tabularx}{\textwidth}{l >{\centering}m{2em} X}
    \emph{Name}     &:& Lintang Al Hilal Fitri \\
    \emph{Title}    &:&	\emph{"Conceptual Design of a Fuel Transport Vessel for the 3T Archipelago Areas: Case Study of Southwest Maluku Regency"} \\
    \emph{Advisors} &:& 1. Muhammad Riduwan, S.Kom., M.Kom  \\
                    & & 2. Ir. Tri Achmadi, Ph.D. \\
  \end{tabularx}
\endgroup


This research aims to address the scarcity of fuel in Southwest Maluku Regency (MBD), a remote archipelagic region of Indonesia. Facing complex distribution challenges, this study will explore solutions through the design of efficient marine transportation operations and modes. The research will identify operational patterns for fuel distribution and design a conceptual model of a marine transportation mode suitable for the environmental conditions of MBD. Additionally, the study will discuss the optimal placement of additional storage to improve the reliability of fuel distribution. The expected outcome of this research is to provide a foundation for enhancing fuel availability and marine transportation infrastructure, thereby supporting sustainable economic growth in Southwest Maluku Regency.


\emph{Keywords}: \emph{Infrastucture}, \emph{Fuel Distribution}, \emph{Marine Transportation}.
